\chapter{introdução}
\label{cap1}

Este trabalho está sendo feito para o curso de "Processamento de sinais digitais".  O assunto proposto foi a multi convolução da função retangular.
O objetivo principal do trabalho é avaliar como a auto convolução \citep{Lourenildo(2015)} vai afetar o formato das funções utilizadas. Assuntos importantes para o desenvolvimento deste trabalho foram: convolução \citep{snieder1998}, transformada de Fourier \citep{Smirnov(1964)} e amplitude espectral \citep{Leite(2007)}.

A ideia geral consiste em utilizar a função retangular e aplicar a convolução com ela mesma, sendo esta a auto convolução \citep{snieder1998}. Este processo será repetido 5 vezes, considerando os resultados de cada um desses processos. O intuito é saber como ficará o formato das funções após cada convolução e se tem um padrão com o aumento destes cálculos. Em seguida, será feito uma análise espectral para comparar as modificações, a partir da amplitude \citep{Lourenildo(2015)}.

Após apresentação da teoria e da metodologia, serão discutidos os resultados obtidos e os significados de cada um dos testes feitos.  








