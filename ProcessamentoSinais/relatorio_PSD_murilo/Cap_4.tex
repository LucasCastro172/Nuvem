\chapter{RESULTADOS E DISCURSÕES}
\label{cap4}

Nesse trabalho geramos um modelo de camadas curvas representando uma estrutura geológica simples (anticlinal e sinclinal) em seguida o sismograma sintético da seção afastamento mínimo. A partir do sismograma sintético, analisamos uma seção tiro. O sismograma da seção tiro 50 gerado no SU serviu de base para realizarmos uma análise tempo-frequência do dado e para o estudo de filtragem F-K.

A análise tempo-frequência  a partir da transformada de Fourier é uma forma efetiva e didática de estudar a relação entre os domínios tempo-frequência e a sua relação linear dos eventos sísmicos baseados na relação da disperção, caracterizada no conteúdo espectral da seção F-K. O método de filtragem F-K mostrou-se eficiente na remoção do ruído aletório, recuperando os eventos da seção perdidos após a soma da seção tiro 50 com o ruído. A aplicação de ganho realçou todos os eventos de maior tempo na seção tiro 50 e consequentemente o ruído foi amplificado, tanto a informação (desejável) quanto o ruído (indesejável).

Um trabalho a ser realizado é melhorar o desenho de filtros na frequência para obter uma melhor resposta da seção filtrada em comparação a seção com ruído. Além disso, uma outra possível implementação visada é construir outros modelos de estruturas geológicas simples e análisar a resposta espectral.