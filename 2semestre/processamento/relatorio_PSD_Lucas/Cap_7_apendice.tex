\appendix
% \addcontentsline{doc}{chapter}{Ap\^endices}
\label{apendice}

\chapter{SHELL SCRIPT MODELAGEM TRIMODEL}
\label{apendice_A}

Este shell script foi escrito para utilizar a subrotina \textit{trimodel} do pacote \textit{Seismic Unix} (SU). 



\lstinputlisting[language=bash,breaklines=true]{codes/model.sh}

\chapter{SHELL SCRIPT AQUISIÇÃO TRISEIS}
\label{apendice_B}

Este shell script foi escrito para utilizar a subrotina \textit{triseis} do pacote \textit{Seismic Unix} (SU). 
O script gera a seção (binário) para o processamento.

\lstinputlisting[language=bash,breaklines=true]{codes/aqsreal.sh}

\chapter{SHELL SCRIPT CDP}
\label{apendice_C}

Este shell script foi escrito para fazer o \textit{plot} das imagens dos \textit{shots} para verificar se aquisição está correta.

\lstinputlisting[language=bash,breaklines=true]{codes/cdp.sh}

\chapter{SHELL SCRIPT RUIDO DE GAUSS}
\label{apendice_D}

Este shell script foi escrito para adicionar um ruído do tipo Gauss a seção sísmica.

\lstinputlisting[language=bash,breaklines=true]{codes/cdp.sh}

\chapter{SHELL SCRIPT ANÁLISE DE VELOCIDADE}
\label{apendice_E}

O objetivo deste programa é de fazer análise de velocidades. 

\lstinputlisting[language=bash,breaklines=true]{codes/velan.sh}

\chapter{SHELL SCRIPT CORREÇÃO NMO}
\label{apendice_F}

O objetivo deste shell script é de fazer a correção NMO nos dados CMP. 

\lstinputlisting[language=bash,breaklines=true]{codes/velan.sh}

\chapter{SHELL SCRIPT EMPILHAMENTO}
\label{apendice_G}

O objetivo deste shell script é de fazer o empilhamento dos dados. 

\lstinputlisting[language=bash,breaklines=true]{codes/stack.sh}

\chapter{SHELL SCRIPT MIGRAÇÃO NO TEMPO}
\label{apendice_H}

O objetivo deste shell script é de fazer a migração temporal dos dados empilhados. 

\lstinputlisting[language=bash,breaklines=true]{codes/migration.sh}






