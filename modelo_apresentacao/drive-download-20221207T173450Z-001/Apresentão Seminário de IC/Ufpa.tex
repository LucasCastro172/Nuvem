%\documentclass[slidestop,usepdftitle=false]{beamer}
\documentclass[blue]{beamer}
\usepackage[accumulated]{beamerseminar}
\usepackage{beamertexpower}
%\usepackage[brazil]{babel}
\usepackage[utf8]{inputenc}
\usepackage[portuges]{babel}
\usepackage{stmaryrd} %produto kulkarni-nomizu
\usepackage[normalem]{ulem}%para tachar/riscar o texto \sout{texto}
\usepackage{graphicx}
\usepackage{graphicx,color}
\usepackage[all]{xy}
%\usepackage[latin1]{inputenc}
\usepackage{amssymb}
\usetheme{Darmstadt}\setbeamercolor{normal text}{}
%\usetheme{Hannover}\setbeamercolor{normal text}{bg=green!10}
\title[UFPA]{Métricas críticas do funcional volume sobre variedades compactas com bordo}
\author{Adam Oliveira da Silva}
\institute{Universidade Federal do Pará - UFPA}
\usetheme{Warsaw}
\newtheorem{proposition}{Proposição}
\newtheorem{cor}{Corolário}
\newtheorem{deff}{Definição}
\newtheorem{thm}{Teorema}
\newtheorem{lem}{Lemma}
\newtheorem{ex}{Example}
%\newtheorem{remark}{Remark}
\newtheorem{remark}{Observação}
\newtheorem{conjec}{Conjectura}
\begin{document}
\frame{\titlepage}



%$$$$$$$$$$$$$$$$$$$$$ PRIMEIRA LAMINA $$$$$$$$$$$$$$$$$$$$$$$$$$$$$$$$$$$
%$$$$$$$$$$$$$$$$$$$$$$$$$$$$$$$$$$$$$$$$$$$$$$$$$$$$$$$$$$$$$$$$$$$$$$$$$$
\begin{frame}
\centerslidesfalse \frametitle{Organização da Palestra}\pause

%\hspace{0.5cm} O trabalho está composto como segue:\pause

\begin{itemize}
	\item Motivações
	\item Exemplos
	\item Alguns resultados de Rigidez
	\item Estimativas de volume
\end{itemize}


\end{frame}



%$$$$$$$$$$$$$$$$$$$$$$$$$$$$$$$$$$$$$$$$$$$$$$$$$$$$$$$$$$$$$$$$$$$$$$$$$
%$$$$$$$$$$$$$$$$$$$$$$$$$$$$$$$$$$$$$$$$$$$$$$$$$$$$$$$$$$$$$$$$$$$$$$$$$





%$$$$$$$$$$$$$$$$$$$$$ SEGUNDA LAMINA $$$$$$$$$$$$$$$$$$$$$$$$$$$$$$$$$$$
%$$$$$$$$$$$$$$$$$$$$$$$$$$$$$$$$$$$$$$$$$$$$$$$$$$$$$$$$$$$$$$$$$$$$$$$$$
%\begin{frame}

%\begin{enumerate}

%\item[3-] Métricas Estáticas\pause
%\begin{itemize}
%	\item Estimativas de volume e resultados de rigidez.\pause
%\end{itemize}
%\end{enumerate}

%\begin{block}{Artigo originado deste estudo}
%\checkmark  Rigidity for critical metrics of the volume functional (arXiv: 1706.07367v1 [math.DG])
%\end{block}

%\end{frame}
%$$$$$$$$$$$$$$$$$$$$$$$$$$$$$$$$$$$$$$$$$$$$$$$$$$$$$$$$$$$$$$$$$$$$$$$$$
%$$$$$$$$$$$$$$$$$$$$$$$$$$$$$$$$$$$$$$$$$$$$$$$$$$$$$$$$$$$$$$$$$$$$$$$$$




%%%%%%%%%%%%%

\begin{frame}
\centerslidesfalse \frametitle{Considerações iniciais}
\pause
Consideremos:
\begin{itemize}
	\item $(M^n, g)$ uma variedade Riemanniana conexa de dimensão $n\geq 3$;\pause
	\item $\mathcal{M}$  o espaço das métricas Riemannianas suaves sobre $M$;\pause
	\item $\mathcal{G}$ o grupo de difeomorfismos sobre $M$.\pause
\end{itemize}
\begin{block}{Definição:}
Chamamos um funcional $\mathcal{F}: \mathcal{M} \rightarrow \mathbb{R}$ de Funcional Riemanniano, se ele é invariante pela ação do grupo $\mathcal{G}$, isto é, se $\mathcal{F}(\varphi^{\ast}g)= \mathcal{F}(g)$ para cada $\varphi\in \mathcal{G}$ e $g\in \mathcal{M}$.
\end{block}
\end{frame}


%%%%%%%%%%%%%%%%%%%%%%%%%

\begin{frame}{Gradiente de funcionais Riemannianos}\pause
\begin{definition}
	Um funcional Riemanniano $\mathcal{F}$ possui um gradiente em $g$, se existe $a\in \Gamma(S^{2}(T^{\ast}M))$ tal que para todo  $h\in \Gamma(S^{2}(T^{\ast}M))$,
	$$\frac{\partial}{\partial t}\mathcal{F}(g(t))\Big|_{t=0} =\mathcal{F}'_{g}(h)=\langle a, h\rangle_{L^2}.$$
\end{definition}\pause
Neste caso, dizemos que $a$ é o gradiente de $\mathcal{F}$ e denotaremos por $a=\nabla \mathcal{F}$.
\end{frame}

\begin{frame}
O \textbf{tensor curvatura de Riemann} é o $(1,3)-$tensor $ Rm:\mathfrak{X}(M)^3\rightarrow\mathfrak{X}(M)$ definido por
\begin{eqnarray*}
	Rm\,(X,Y)Z&=&\nabla_{X,Y}^{2}Z-\nabla_{Y,Z}^{2}Z\\
	&=&\nabla_{X}\nabla_{Y}Z-\nabla_{Y}\nabla_{X}Z-\nabla_{[X,Y]}Z,
\end{eqnarray*}
para todo $X,Y,Z\in\mathfrak{X}(M).$\pause

\begin{equation*}
Rm\,(X,Y,Z,W)=-\langle Rm\,(X,Y)Z,W\rangle.
\end{equation*}\pause

Em coordenadas:
\begin{eqnarray*}
	Rm(\partial_{i}, \partial_{j})\partial_{k} &=&{R_{ijk}}^{l}\partial_{l}\\
	Rm(\partial_{i}, \partial_{j}, \partial_{k}, \partial_{l}) &=&R_{ijkl}. 
\end{eqnarray*}
Assim,
\begin{equation*}
R_{ijkl} = - \langle Rm(\partial_{i}, \partial_{j})\partial_{k} , \partial_{l}\rangle = -\langle {R_{ijk}}^{m}\partial_{m}, \partial_{l}\rangle = -{R_{ijk}}^{m}g_{ml}, 
\end{equation*}
isto é, o índice superior desce na terceira posição. 
\end{frame}

\begin{frame}
Dado um plano bidimensional $\Pi\subset T_{p}M$ e $X_{p}, Y_{p}\in T_{p}M$ vetores que geram $\Pi$, então
\begin{equation}
K(\Pi)=\frac{Rm(X, Y, X, Y)}{g(X, X)g(Y, Y)-g(X, Y)^{2}},
\end{equation}
não depende da base escolhida para $\Pi$, e é chamada {\bf curvatura seccional} do plano $\Pi$. \pause 

Uma variedade Riemanniana completa e com curvatura seccional constante é dita uma {\bf forma espacial}.
\end{frame}

\begin{frame}
O \textbf{tensor de Ricci} é definido como o $(0,2)-$ tensor
\begin{equation*}
{\rm Ric}(X,Y)= tr(U \rightarrow {\rm Rm}(U,X)Y).
\end{equation*}
Em coordenadas teremos:
\begin{equation*}
R_{ij} = {R_{lij}}^{l}= g^{lm}R_{limj}
\end{equation*}
e a \textbf{curvatura escalar} é
\begin{equation*}
R = g^{ij}R_{ij}.
\end{equation*}
\end{frame}


%%%%%%%%%%%%%%%%%%%%%%%%%
\begin{frame}{Motivação: Comparação de Volume}\pause 
\begin{block}{(Bishop - Gromov, 1964)}
	Seja $(M^n, g)$ uma variedade Riemanniana completa com $Ric \geq (n-1)kg$, $k$ constante, e $p\in M$ um ponto arbitrário. Então
	$$Vol (B_{r}(p)) \leq Vol(B^{k}_{r}).$$ 
\end{block}\pause

Pergunta: \pause
\begin{itemize}
	\item \textcolor{blue}{controle na curvatura escalar}$ \Rightarrow $ \textcolor{red}{Comparação de volume ? 	}\pause
\end{itemize}

\begin{block}{Conjectura: (Schoen, 1989)}
	Seja $(M^n, g)$ uma variedade hiperbólica fechada. Se $h$ é outra métrica sobre $M$ com $R_{h}\geq R_{g}$, então $Vol(h) \geq Vol(g)$.
\end{block}

\end{frame}

\begin{frame}{(Miao -Tam, 2009)}\pause
\begin{itemize}
\item$(M^3, g)$ uma variedade Riemanniana com bordo $\partial M = \Sigma$.\pause
\item $(\Sigma, \gamma)$ isometricamente mergulhada em $\mathbb{R}^{3}$ como uma hipersuperfície estritamente convexa $\Sigma_{0}$.\pause
\item $g$ ponto crítico do funcional volume $V(.)$ sobre $\mathcal{M}_{\gamma}^{0} = \{g, R_{g} = 0 \; \mbox{e} \; g|_{\Sigma}= \gamma\}$.\pause
\end{itemize}

Então
$$V_{g}\geq V_{0},$$
com igualdade $\Leftrightarrow$ $(M, g)$ é isométrica à bola Euclidiana padrão.

\end{frame}




%%%%%%%%%%%%%
\begin{frame}{Motivação: Caracterização variacional}\pause
Hilbert (1915): \pause
\begin{block}{Funcional curvatura escalar total ou Funcional de Einstein-Hilbert}
$$g \rightarrow \int_{M}R_{g} dV_{g},$$
onde $R_{g}$ e $dV_{g}$ denotam, respectivamente, a curvatura escalar e a forma de volume de  $M^n$.
\end{block}\pause

\vspace{0.2cm}
{\bf Relatividade Geral:} As equações de Einstein surgem como as equações de Euler-Lagrange desse funcional.
\end{frame}








%%%%%%%%%%%%%%%%%%%%%%

\begin{frame}
$$\mathcal{M}_{1}=\{g\in \mathcal{M} | Vol(g)=1\}$$\pause
\begin{block}{(Hilbert, 1915)}
Uma métrica $g\in\mathcal{M}_{1}$ é ponto crítico para o funcional de Einstein-Hilbert se, e somente se, $g$ é uma métrica de Einstein, isto é, $Ric_{g}=\frac{R_{g}}{n}g$.
\end{block}\pause

\vspace{0.2cm}
Problema de Yamabe:\pause
\begin{block}{(Schoen, 1984)}
Dada uma variedade Riemanniana $(M,g)$, o funcional $$g \rightarrow Vol(g)^{-(n-2)/n}\int_{M}R_{g} dV_{g}$$ atinge um mínimo na classe conforme de $g$.
\end{block}
\end{frame}

%%%%%%%%%%%%%%%%%%%%%%%
%\begin{frame}
%Relembremos que o tensor de Riemann admite a seguinte decomposição:\pause
%\begin{eqnarray}
%\label{weyl}
%R_{ijkl}&=&W_{ijkl}+\frac{1}{n-2}\big(R_{ik}g_{jl}+R_{jl}g_{ik}-R_{il}g_{jk}-R_{jk}g_{il}\big) \nonumber\\
%&&-\frac{R}{(n-1)(n-2)}\big(g_{jl}g_{ik}-g_{il}g_{jk}\big), 
%\end{eqnarray}
%de onde segue-se que\pause
%\begin{eqnarray*}\label{Rie}
%	\mathcal{R}(g)&=&\int_{M} |Rm_{g}|^2 dV_{g}\\
%	& =& \int_{M} \left(|W_{g}|^2 + \dfrac{4}{n-2}|Ric_{g}|^2 - \dfrac{2}{(n-1)(n-2)}R_{g}^2\right)dV_{g}.
%\end{eqnarray*}
%\end{frame}

%%%%%%%%%%%%%%%%%%%%




%%%%%%%%%%%%%%%%%%%%%%%%%

\begin{frame}
Consideremos:\pause
\begin{itemize}
	\item $(M^n, g)$ uma variedade Riemanniana conexa, compacta e com bordo conexo suave $\Sigma$, $n\geq 3.$\pause
	\item $\gamma$ uma métrica suave sobre $\Sigma$.\pause
	\item $\mathcal{M}^{R}_{\gamma}=\{g\in \mathcal{M}; R_{g}=R \;\;\mbox{e}\;\; g|_{T\Sigma}=\gamma \}.$ \pause
	\item O funcional volume $V : \mathcal{M}^{R}_{\gamma} \rightarrow \mathbb{R}$ dado por
	$$V(g)=\int_{M}dV_{g}.$$
\end{itemize}
\end{frame}

\begin{frame}{Caracterização variacional}\pause
\begin{thm}[Miao e Tam, 2009]
	Seja $g\in\mathcal{M}_\gamma^{R}$ tal que o primeiro autovalor de Dirichlet de $(n-1)\Delta_{g}+R$ é positivo.
	Então $g$ é ponto crítico do funcional volume $V(\cdot)$ em $\mathcal{M}_\gamma^{R}$ se, e somente se, existe
	uma função $f$ em $M$ satisfazendo o seguinte sistema de Equações Diferenciais Parciais
	\begin{equation*}\label{teomotiveq1}
	\left\{\begin{array}{rr}
	-(\Delta_g f)g+\nabla_g^2f-fRic_g=g, &\mbox{ em } M\\
	f=0, &\mbox{ sobre } \Sigma. \end{array} \right.
	\end{equation*}
\end{thm}\pause
\begin{remark}
	Se $(M,g)$ satisfaz $-(\Delta_g f)g+\nabla_g^2f-fRic_g=g$, então $R_{g}$ é constante.
\end{remark}


\end{frame}



\begin{frame}

\begin{deff}\
Uma \textbf{métrica crítica de Miao-Tam}, ou simplesmente \textbf{métrica crítica}, é uma tripla $(M^{n},\,g,\,f)$, $n\geq 3$, onde $(M^n,\,g)$ é uma variedade
Riemanniana compacta e conexa com bordo suave $\partial M=\Sigma$ e $f$ é uma função suave em $M$ tal que
$f^{-1}(0)=\Sigma$ e satisfaz ao seguinte sistema de equações:
\begin{equation*}
\label{eqMiaoTam}-(\Delta_{g} f)g+\nabla_{g}^2f-fRic_{g}=g,
\end{equation*}
onde $\nabla_{g}^2 f$ denota o Hessiano de $f$. Tal função $f$ será chamada de função potencial.
\end{deff}\pause

\begin{block}{(Miao e Tam, 2009)}
$g$ é ponto crítico de $V$ $\Leftrightarrow$ $g$ é uma métrica de Miao-Tam.
\end{block}

\end{frame}

%%%%%%%%%%%%%%%%%%%%%
\begin{frame}{Exemplos de métricas de Miao-Tam}\pause
\begin{block}{Bola geodésica em $\mathbb{R}^{n}$}
\begin{itemize}
\item $(M^n, g)$ bola geodésica centrada na origem de raio $R_{0}$ em $\mathbb{R}^{n};$
\item $f(x) = \frac{1}{2(n-1)}(R_{0}^{2}-|x|^{2}).$
\end{itemize}
\end{block}\pause
\begin{block}{Bola geodésica em $\mathbb{H}^{n}$}
\begin{itemize}
\item $(M^n, g)$ bola geodésica centrada em $p\in \mathbb{H}^n$ de raio $R_{0};$
\item $f(x) = \frac{1}{(n-1)}(1-\frac{\cosh r(x)}{\cosh R_{0}}).$
\end{itemize}
\end{block}\pause

\begin{block}{Bola geodésica em $\mathbb{S}^{n}$}
\begin{itemize}
\item $(M^n, g)$ bola geodésica centrada em $p\in \mathbb{S}^n$ de raio $R_{0}<\frac{\pi}{2};$
\item $f(x) = \frac{1}{(n-1)}(\frac{\cos r(x)}{\cos R_{0}}-1).$
\end{itemize}
\end{block}
\end{frame}

%%%%%%%%%%%%%%%%%%%%%%
\begin{frame}{Alguns resultados existentes}\pause
\begin{block}{(Miao e Tam, 2009)}
Se $M$ é um domínio limitado com bordo suave em $\mathbb{R}^n,$ $\mathbb{H}^n$ ou $\mathbb{S}^n$ $($se
$M^n\subset\mathbb{S}^n,$ suponha ainda que $V(M)<\frac{1}{2}V(\mathbb{S}^n)).$ Então a correspondente métrica
nesse espaço é um ponto crítico do funcional volume $V(\cdot)$ em $\mathcal{M}_\gamma^{R}$ se, e somente se, $M$
é uma bola geodésica.
\end{block}\pause
\begin{block}{Questão}
As bolas geodésicas das formas espaciais simplesmente conexas $\mathbb{R}^n$, $\mathbb{S}^{n}$ e $\mathbb{H}^n$ são as únicas métricas críticas de Miao-Tam?\pause
\\\textcolor{red}{Não!!!}
\end{block}\pause
\begin{block}{(Miao e Tam, 2011)}
Construiram exemplos de métricas críticas conformemente planas que não são  métricas de Einstein.	
\end{block}
\end{frame}
%%%%%%%%%%%%%%%%%%%%%%%%%%

\begin{frame}

\begin{block}{(Miao e Tam, 2011)}
	Seja $(M^n,\,g,\,f)$ uma métrica crítica de Miao-Tam localmente conformemente plana, simplesmente conexa e com bordo $\Sigma$ isométrico à esfera canônica $\mathbb{S}^{n-1}$. Então $(M^n,\,g)$ é isométrica a uma bola geodésica em $\mathbb{R}^n,$ $\mathbb{H}^n$ ou $\mathbb{S}^n.$
\end{block}\pause 

\begin{block}{(Miao e Tam, 2011)}
Seja $(M^n,\,g,\,f)$ uma métrica crítica de Miao-Tam Einstein e bordo $\Sigma$ suave. Então $(M^n,\,g)$ é isométrica a uma bola geodésica em $\mathbb{R}^n,$ $\mathbb{H}^n$ ou $\mathbb{S}^n.$
\end{block}

\end{frame}


\begin{frame}
\begin{block}{Classificação de métricas críticas de Miao-Tam}
	\begin{itemize}
			\item (Barros-Diógenes-Ribeiro Jr, 2015)\\
		\checkmark \textcolor{red}{n=4, simp. conexa, Bach-flat e $\Sigma \approx \mathbb{S}^{3}$}\pause
	
		\item (Kim-Shin, 2016)\\
	\checkmark \textcolor{red}{n=4, simp. conexa, $divW = 0$ e $\Sigma \approx \mathbb{S}^{3}$ }\pause
	
		\item (Baltazar-Ribeiro Jr., 2017)\\
		\checkmark \textcolor{red}{Ricci Paralelo}
	
		%\item (Baltazar-Ribeiro Jr., 2017)\\
		%\checkmark \textcolor{red}{n=4, simp. conexa, $div^2W = 0$ e $\Sigma \approx \mathbb{S}^{3}$ }
	\end{itemize}\pause
	
\end{block}
\begin{block}{}
	Em qualquer caso temos que $M^n$ é isométrica a uma bola geodésica em $\mathbb{R}^n,$ $\mathbb{H}^n$ ou $\mathbb{S}^n.$
\end{block}
\end{frame}





%%%%%%%%%%%%%%%%%%

%%%%%%%%%%%%%%%%%%%%%%%


%%%%%%%%%%%%%%%%


%%%%%%%%%%%%%%%%%%%%%%%%

%%%%%%%%%%%%%%%%%%%%%%%%%%%%%
\begin{frame}
\begin{block}{(Batista-Diógenes-Raniere-Ribeiro Jr., 2016)}
Seja $(M^3, g, f)$ uma métrica crítica de Miao-Tam, compacta, orientada, com bordo conexo $\Sigma$ e curvatura escalar não negativa. Então, $\Sigma$ é uma esfera bidimensional e 
\begin{equation}\label{estMarcos}
|\Sigma|\leq \frac{4\pi}{C(R)},
\end{equation}
onde $C(R)= \frac{R}{6} +\frac{1}{4|\nabla f|^2}$ é constante. Além disso, a igualdade em $(\ref{estMarcos})$ ocorre se, e somente se, $(M^3, g)$ é isométrica a bola geodésica em alguma forma espacial simplesmente conexa $\mathbb{R}^3$ ou $\mathbb{S}^3$.
\end{block}\pause

\begin{block}{Observação}
Ainda em $2016$, E. Barbosa et al. mostraram que este resultado
 também é \textcolor{red}{válido} no caso de \textcolor{red}{curvatura escalar negativa}, supondo a curvatura média do bordo \textcolor{red}{$H>2$}.
 %\pause Além disso, provaram um resultado semelhante para dimensão \textcolor{red}{$n=5$} desde que o bordo \textcolor{red}{$\Sigma^{4}$ seja de Einstein}.
\end{block}

\end{frame}


\begin{frame}{Estimativas e resultados de Rigidez}
\begin{thm}[Barros, ---, 2017]
	Seja $(M^n, g, f), \,n\geq 4,$ uma métrica crítica de Miao-Tam, compacta, orientada, com bordo conexo $\Sigma$ e curvatura escalar $R=n(n-1)\varepsilon $, onde $\varepsilon = -1, 0, 1$. Suponha que $\Sigma$ seja uma variedade de Einstein com curvatura escalar $R^{\Sigma}$ positiva. Se $\varepsilon = -1,$ assumimos ainda que a curvatura média de $\Sigma$ satisfaz $H> n-1$. Então temos
	\begin{equation}\label{boundM}
	|\Sigma|^{\frac{2}{n-1}} \leq \frac{Y(\mathbb{S}^{n-1}, [g_{can}])}{C(R)},
	\end{equation}
	onde $C(R)=\frac{n-2}{n}R+\frac{n-2}{n-1}H^{2}$ é uma constante positiva. Além disso, a igualdade ocorre em $(\ref{boundM})$ se, e somente se, $(M^n, g)$ é isométrica a uma bola geodésica em alguma das formas espaciais simplesmente conexas $\mathbb{S}^n$, $\mathbb{R}^{n}$ ou $\mathbb{H}^n$.
\end{thm}
\end{frame}



%%%%%%%%%%%%%%%%%%%%%%%%%%%


\begin{frame}{Considerações}\pause
\begin{itemize}
	\item $(M^n, g)$ variedade Riemanniana fechada de dimensão $n\geq 3;$\pause
	\item $[g]$ a classe conforme de uma métrica $g\in \mathcal{M};$\pause
	\item Constante de Yamabe: $$Y(M, [g]) = \inf_{\tilde{g}\in [g]} \frac{\int_{M} R_{\tilde{g}}dV_{\tilde{g}}} {(\int_{M}dV_{\tilde{g}})^{\frac{n-2}{n}}}.$$\pause
	\item $Y(\mathbb{S}^{n}, [g_{can}]) = n(n-1)\omega_{n}^{2/n},$ onde $\omega_{n}$ denota o volume da esfera canônica unitária $\mathbb{S}^{n}$.
\end{itemize}

\end{frame}
%%%%%%%%%%%%%%%%%%%%%%%%%%






\begin{frame}{Fatos sobre métricas críticas de Miao-Tam}\pause
	\begin{equation*}\label{Dirichlet}
\begin{cases}
-(\Delta_{g} f)g+\nabla_{g}^2f-fRic_{g}=g & \mbox{em}  \quad M\\
f=0 & \mbox{sobre}  \quad \Sigma.
\end{cases}
\end{equation*}\pause

\begin{itemize}
	\item Curvatura escalar constante $R_{g}=n(n-1)\varepsilon$, onde $\varepsilon=-1,0,1$;\pause
	\item $|\nabla f|$ é constante positiva sobre $\Sigma$;\pause
	\item $\Sigma$ é uma hipersuperfície totalmente umbílica com curvatura média $H=\frac{1}{|\nabla f|}$;\pause
	\item $2Ric(\nu, \nu) + R^{\Sigma} = R+ \frac{n-2}{n-1}H^2,$ onde $\nu =- \frac{\nabla f}{|\nabla f|}$ é o campo normal unitário exterior ao bordo $\Sigma$.
\end{itemize}
\end{frame}

%%%%%%%%%%%%%%%%%%%%%%%%%%%%%%%%%

\begin{frame}

$$f\mathring{Ric}=\mathring{\nabla^{2}f}$$

\begin{equation*}
\label{divric}
f|\mathring{Ric}|^2 = \langle \mathring{Ric}, \mathring{\nabla^2 f}\rangle = div(\mathring{Ric}(\nabla f)).
\end{equation*}
\begin{lemma}Seja $(M^n ,g, f)$ uma métrica crítica de Miao-Tam compacta, orientada, conexa e com bordo suave conexo $\Sigma$. Então,
	\begin{equation*}
	\int_{M}f|\mathring{Ric}|^2 dV_{g} = -H\int_{\Sigma}\mathring{Ric}(\nabla f, \nabla f) ds.
	\end{equation*}
\end{lemma}
\end{frame}

\begin{frame}
\begin{proposition}\label{intRbound}
	Seja $(M^n, g, f)$, $n\geq 3$, uma métrica crítica de Miao-Tam compacta, orientada, conexa, com bordo suave conexo $\Sigma$ e curvatura escalar $R= n(n-1)\varepsilon$, onde $\varepsilon = -1, 0, 1$. Então, a seguinte identidade ocorre
	\begin{equation*}
	\int_{\Sigma} R^{\Sigma} ds = 2H \int_{M} f|\mathring{Ric}|^2 dV_{g} + C(R)|\Sigma|,
	\end{equation*}onde $C(R)$ é uma constante dada por $$C(R)=\frac{n-2}{n-1}(H^2 + (n-1)^2\varepsilon).$$
	%	\begin{equation*}
	% \int_{\Sigma} R^{\Sigma} ds = 2H \int_{M} f|\mathring{Ric}|^2 dV_{g} + \frac{n-2}{n-1}\big(H^2+\varepsilon (n-1)^2\big)|\Sigma|.
	%\end{equation*}
\end{proposition}
\end{frame}



%%%%%%%%%%%%%%%%%%%%%



%%%%%%%%%%%%%%%%%%%%%%%%%%%%%%%

\begin{frame}
\begin{block}{(Ilias, 1983)}
Seja $(M^n , g),\,n\ge 3,$ uma variedade Riemanniana compacta sem bordo. Suponha que $\mathcal{R}(M, g)\geq \mathcal{R}(\mathbb{S}^n , \frac{1}{\delta}g_{can})=(n-1)\delta>0$, então
\begin{eqnarray*}
\Big(\int_{M} |f|^{\frac{2n}{n-2}} dV_{g}\Big)^{\frac{n-2}{n}}&\leq &[K(n,2)]^2 \Big(\frac{\omega_{n}(\delta)}{|M|}\Big)^{\frac{2}{n}} \int_{M}|\nabla f|^2 dV_{g}\\
& +& |M|^{-\frac{2}{n}}\int_{M} |f|^2 dV_{g},
\end{eqnarray*}
para toda $f\in H^{1,2}(M)$, onde $\omega_{n}(\delta)=\delta^{-\frac{n}{2}}\omega_{n}$.
\end{block}

\end{frame}

%%%%%%%%%%%%%%%%%%%%%%%%5

\begin{frame}
\begin{block}{}
	$$	\label{infric}\mathcal{R}(M, g) = \inf\{Ric(V, V) \;|\; V\in TM, |V|_{g}=1\};$$
	$$K(n,2)= \sqrt{\frac{4}{n(n-2)\omega_{n}^{2/n}}}$$ é a melhor constante para desigualdades do tipo Sobolev:
	\begin{equation}\label{sob1}
	\Big(\int_{\Sigma} |\varphi|^{p} ds\Big)^{\frac{1}{p}}\leq A \Big(\int_{\Sigma}|\nabla \varphi|^{q} ds\Big)^{\frac{1}{q}} + B\Big(\int_{\Sigma} |\varphi|^{q} ds\Big)^{\frac{1}{q}},
	\end{equation}onde $\frac{1}{p}=\frac{1}{q}-\frac{1}{n-1}$, $1\leq q <n-1$ e $q\in\mathbb{R}.$
\end{block}
\end{frame}


\begin{frame}
Aplicando à variedade $\Sigma^{n-1}$ o teorema citado devido à Ilias para  $\delta = \frac{R^{\Sigma}}{(n-1)(n-2)}>0$, obtemos
\begin{eqnarray}
\nonumber\Big(\int_{\Sigma} |\varphi|^{\frac{2(n-1)}{n-3}} ds\Big)^{\frac{n-3}{n-1}}&\leq& [K(n-1,2)]^2 \Big(\frac{\omega_{n-1}(\delta)}{|\Sigma|}\Big)^{\frac{2}{n-1}} \int_{\Sigma}|\nabla \varphi|^2 ds \\
\nonumber &+& |\Sigma|^{-\frac{2}{n-1}}\int_{\Sigma} |\varphi|^2 ds,
\end{eqnarray} para toda $\varphi\in H^{1,2}(\Sigma)$, onde $\omega_{n-1}(\delta)=\delta^{-\frac{n-1}{2}}\omega_{n-1}$, $\omega_{n-1} = |\mathbb{S}^{n-1}|$ e $K(n-1,2)$ é a melhor constante para desigualdades do tipo Sobolev.
\end{frame}

\begin{frame}
\begin{eqnarray*}
	\Big(\int_{\Sigma} |\varphi|^{\frac{2(n-1)}{n-3}} ds\Big)^{\frac{n-3}{2(n-1)}}&\leq &[K(n-1,2)] \Big(\frac{\omega_{n-1}(\delta)}{|\Sigma|}\Big)^{\frac{1}{n-1}}\Big( \int_{\Sigma}|\nabla \varphi|^2 ds\Big)^{\frac{1}{2}}\\
	&+& |\Sigma|^{-\frac{1}{n-1}}\Big(\int_{\Sigma} |\varphi|^2 ds\Big)^{\frac{1}{2}}.
\end{eqnarray*}\pause

Com isto, temos
	\begin{equation*}\label{ineqwsigma}
(\omega_{n-1})^{\frac{2}{n-1}} \geq \delta |\Sigma|^{\frac{2}{n-1}}.
\end{equation*}\pause 

Substintuindo  $\delta = \frac{R^{\Sigma}}{(n-1)(n-2)}>0$, obtemos
\begin{eqnarray}\label{ineqYam}
Y(\mathbb{S}^{n-1}, [g_{can}]) \geq R^{\Sigma} |\Sigma|^{\frac{2}{n-1}}.
\end{eqnarray}
\end{frame}

\begin{frame}{Prova}
Integrando a equação anterior sobre $\Sigma$ e usando a Proposição acima, obtemos
\begin{equation}\label{ineqSigY}
|\Sigma|^{\frac{2}{n-1}} \leq \frac{Y(\mathbb{S}^{n-1}, [g_{can}])}{C(R)}.
\end{equation} \pause
Além disso, se ocorre a igualdade em \eqref{ineqSigY} devemos ter
$$\int_{M} f|\mathring{Ric}|^2 dV_{g}=0.$$ Isto é, $(M^n,g)$ é uma variedade de Einstein.\pause  
Logo, temos que $M^n$ é isométrica a uma bola geodésica em $\mathbb{R}^{n}$, $\mathbb{S}^{n}$ ou $\mathbb{H}^{n}$. \pause

A recíproca?? \pause (Exercício)
\end{frame}



\begin{frame}
\begin{cor}
	Seja $(M^n, g, f), \,n\geq 4,$ uma métrica crítica de Miao-Tam, compacta, orientada, com bordo conexo $\Sigma$ isométrico à esfera canônica $\mathbb{S}^{n-1}(r)$ de raio $r=\Big(\frac{(n-1)(n-2)}{C(R)}\Big)^{1/2}$, e curvatura escalar $R=n(n-1)\varepsilon $, onde $\varepsilon = -1, 0, 1$. Além disso, se $\varepsilon = -1,$ assumimos que a curvatura média de $\Sigma$ satisfaz $H> n-1$. Então $(M^n, g)$ é isométrica a uma bola geodésica em alguma das formas espaciais simplesmente conexas $\mathbb{S}^n$, $\mathbb{R}^{n}$ ou $\mathbb{H}^n$.
\end{cor}\pause
\begin{cor}
	Com as mesmas condições do Teorema, porém $R\ge 0$, deduzimos
	\begin{equation}\label{EstVolM}
	\Big(\frac{nH}{n-1}\Big)^{\frac{2}{n-1}}|M|^{\frac{2}{n-1}}\leq \frac{Y(\mathbb{S}^{n-1}, [g_{can}])}{C(R)}.
	\end{equation}
	Ainda, a igualdade acontece em $(\ref{EstVolM})$ se, e somente se, $(M^n, g)$ é isométrica a uma bola geodésica no espaço euclidiano $\mathbb{R}^{n}$.
\end{cor}
\end{frame}

%%%%%%%%%%%%%%%%%%%%%%%%%%%%%%





%%%%%%%%%%%%%%%%%%%%%%%%%%%%%%%%%%%%%%%%%%%%%%%%%%%%%%%%%%%%%%%%%%%%%%%%%%%%%%%%%%%%%%%%%%%%%%%%%%%%%%%
%%%%%%%%%%%%%%%%%%%%%%%%%%%%%%%%%%%%%%%%%%%%%%%%%%%%%%%%%%%%%%%%%%%%%%%%%%%%%%%%%%%%%%%%%%%%%%%%%%%%%%%%%
%\begin{frame}
%\begin{thebibliography}{BB}
%\centerslidesfalse \frametitle{Refer\^encias}

%\bibitem{Baltazar} Baltazar, H.; Ribeiro Jr.: Critical metrics of the volume functional on manifolds with boundary. Proceedings of the American Mathematical Society, v. 145, p. 3513-3523, 2017.

%\bibitem{Barbosa} Barbosa, E.; Lima, L.; Freitas, A.: The generalized Pohozaev-Schoen identity and some geometric applications. arXiv:1607.03073v1 [math.DG], 2016.

%\bibitem{Batista} Batista, R.; Diógenes, R.; Raniere, M.; Ribeiro Jr.: Critical metrics of the volume functional on compact three-manifolds with smooth boundary. The Journal of Geometric Analysis, v. online, p. 1-18, 2016.

%\bibitem{Berger} Berger, M.: Quelques formules de variation pour une structure riemannienne. Annales Scientifiques de l'É.N.S., v. 3, n. 3, p. 285-294, 1970.




%\bibitem{ranieri} Batista, R.; Diógenes, R.; Ranieri, M; Ribeiro Jr., E.: Critical metric of the volume functional on compact three manifolds with smooth
%boundary. The Journal of Geometric Analysis, v. 27, p. 1530-1547, 2017.




%\end{thebibliography}{BB}
%\end{frame}

%%%%%%%%%%%%%%%%%%%%%%%%%%%%%%%%

%\begin{thebibliography}{BB}

%\bibitem{Besse} Besse, A.: Einstein Manifolds. New York: Springer-Verlag, 1987.

%\bibitem{Bou} Boucher, W.; Gibbons, G.; Horowitz, G.: Uniqueness theorem for anti-de Sitter spacetime. Physical Review D, v. 30, p. 2447-2451, 1984.


%\bibitem{Catino} Catino, G.: Some rigidity results on critical metrics for quadratic functionals. Calculus of Variations, v. 54, p. 2921-2937, 2015.

%\bibitem{Corvino} Corvino, J.: Scalar curvature deformation and a gluing construction for the Einstein constraint equations. Communications in Mathematical Physics, v. 214, n. 1, p. 137-189, 2000.

%\bibitem{Gur} Gursky, M.; Viaclovsky, J.: A new variational characterization of three-dimensional space forms. Inventiones Mathematicae, v. 145, p. 251-278, 2001.

%\bibitem{Hilbert} Hilbert, D.: Die Grundlagen der Physik. Annales Scientifiques de l'É.N.S., v. 4, p. 461-472, 1915.

%\bibitem{Hu} Hu, Z.; Li, H.: A new variational characterization of n-dimensional space forms. Transactions of the American Mathematical Society, v. 356, n. 8, p. 3005-3023, 2003.

%\bibitem{Huang} Huang, G.: Some rigidity characterizations on critical metrics for quadratic curvature functionals. arXiv: 1707.04806v1 [math.DG], 2017.

%\bibitem{Ilias} Ilias, S.: Constantes explicites pour les inégalités de Sobolev sur les variétés riemanniennes compactes. Annales de l'institut Fourier, v. 33, n. 2, p. 151-165, 1983.

%\bibitem{Kob} Kobayashi, O.: A differential equation arising from scalar curvature function. Journal of the Mathematical Society of Japan, v. 34, n. 4, p. 665-675, 1982.

%\bibitem{Lafon} Lafontaine, J.: Sur la géométrie d'une généralisation de l'équation différentielle d'Obata. Journal de Mathématiques Pures et Appliquées, v. 62, n. 1, p. 63-72, 1983.

%\bibitem{miaotam} Miao, P.; Tam, L.-F.: On the volume functional of compact manifolds with boundary with constant scalar curvature. Calculus of Variations and
%Partial Differential Equations, v. 36, n. 2, p. 141-171, 2009.

%\bibitem{miaotrans} Miao, P.; Tam, L.-F. Einstein and conformally at critical metrics of the volume functional. Transactions of the American Mathematical
%Society, v. 363, n. 6, p. 2907-2937, 2011.

%\bibitem{Qing} Qing, J.; Yuan, W.: A note on static spaces and related problems. J. Geom. and Phys., v. 74, p. 18-27, 2013.

%\bibitem{Schoen} Schoen, R.: Conformal deformation of a Riemannian metric to constant scalar curvature. Journal of Differential Geometry, v. 20, n. 2, p. 479-495, 1984.

%\bibitem{Shen} Shen, Y.: A note on Fischer-Marsden's conjecture. Proceedings of the American Mathematical Society, v. 125, n. 3, p. 901-905, 1997.

%\bibitem{Tanno} Tanno, S.: Deformations of Riemannian metrics on 3-dimensional manifolds. Tôhoku Mathematical Journal , v. 27, n. 3, p. 437-444, 1975.



%\bibitem{AA} Anderson, M.: Scalar curvature, metric degenerations and the static vacuum Einstein equations on 3-manifolds. Geom. and Funct. Anal., v. 9, p. 855-967, 1999.

%\bibitem{AC} Bunting, G.L., Masood-ul-Alam, A.K.M.: Non-existence of multiple black holes in asymptotically Euclidean static vacuum space-times.  Glasgow Math. J., v. 19, p. 147-154, 1987.

%\bibitem{CHE} Israel, W.: Event horizons in static vacuum space-time. Phys. Rev., v. 164, p. 1776-1779, 1967.
%p. 139-148, 2014.

%\bibitem{HE} Hwang, S., Chang, J., Yun, G.: Nonexistence of multiple black holes in static space-times and weakly harmonic curvature. Gen. Relativ. Gravit., v. 48, p. 120, 2016.

%%%%%%%%%%%%%%%%%%%%%%%%%%%%%%%%%%%%%%%%%%%%%%%%%
%\end{thebibliography}
\begin{frame}
\resizebox{!}{1cm}{\hspace{.6cm}Obrigado!}
%\begin{figure}[ht]
%	\centering
%  \includegraphics[width=8cm, height=5cm]{RORAIMA.png}
%	%\caption{Jean-Pierre Bourguignon(francês)}
%	\label{fig1}
%\end{figure}\pause
\end{frame}
\end{document}





